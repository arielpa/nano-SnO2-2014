%submission by M.A.Caravaca, R.A.Casali et al.
\documentclass[preprint,aps,prb,amsmath]{revtex4-1}%,prb,,amssymb]{revtex4} 

\usepackage{graphicx}  %[pdftex]
% \DeclareGraphicsExtensions{.png,.pdf,.jpg,.eps}
\usepackage[dvips]{hyperref}
\usepackage[TS1,OT1,T1]{fontenc}
\usepackage{graphics}
\usepackage[latin1]{inputenc}
\usepackage[T1]{fontenc}
%\bibliographystyle{prsty}

%-----------------------------------------------------------------------------------------
\usepackage{dcolumn}  % paquete para alinear separaciones por punto decimal en tablas. Modo de uso:
% D{separador 1}{separador 2}{decimales}
% separador 1: Es el que usaremos en el codigo para separar la parte entera de
% la decimal.
% separador 2: Es lo que LATEXnos mostrara en la salida para separar la parte
% entera de la decimal.
% decimales Es la cantidad de decimales que se mostraran en la salida, si el
% valor es \-1" no se limitara la cantidad de cimales en la salida.
% ejemplo::   \begin{tabular}{|l | D{.}{,}{-1} |}


\begin{document}
\title{Ab-initio studies of relative stability of SnO$_2$ nanoparticles as a function of stoichiometry, partial pressure and oxygen temperature}

\author{C.A. Ponce}
\affiliation{Departamento de Fisica, Facultad de Ciencias Exactas y Nat. y
Agr.-UNNE- Av. Libertad 5600 - C.P.3400 - Corrientes - Argentina}

\author{M.A. Caravaca }
\affiliation{Departamento de Fisico-Quimica, Facultad de
Ingenieria, UNNE - Av. Las Heras 727 - C.P.3500 - Resistencia -
Argentina}

\author{R.A. Casali}
\affiliation{Departamento de Fisica, Facultad de Ciencias Exactas y Nat. y
Agr.-UNNE- Av. Libertad 5600 - C.P.3400 - Corrientes - Argentina}


\date{\today}
\begin{abstract}
    Relative stability of nano-SnO2 in rutlie structure with defined sizes and controlled stoichiometry was studied allowing the 
nanoparticle atoms relax internally. Once stabilized in energy, the structure was analyzed by a refniement of the Pair  distribution
function of the Sn-Sn and Sn-O distances. This function shows that a rutile crystalline core of significant size remains in the NP's
grater than 3nm. Disordered surface layers is reduced to 0.25 nm wide for stoichiometry and the oxygen excess nanoparticles.
The volume of the crystalline core of the nanoparticle, is 50\% greater, compared to the distorted layer volume in nanoparticles
of 3.3 nm size and explain whe the nanoparticles of the size are distinguishable by XRD. THe effects of the PDOS of the
different oxygen defects at the surface, are analyzed. Results of energies of formation, nanoparticles surface energies and the
effects of temperature and oxygen partial pressure in rutile NP are presented.

\end{abstract}

\pacs{62.20.Dc, 61.82.Ms, 77.84.Bw}
\maketitle


\begin{section}{Introduction}

Tin dioxide (oxide of tin or SnO$_{2}$) is a material with important technological applications, such as solar cells, gas sensors and
 optoelectronic devices. It was one of the first oxides considered for and most frequently used in high sensibility gas sensor by 
electrical conductivity variations. \cite{Gopel1995},
The sensitivity of these oxides is closely linked to their chemical surface activities, that is, a larger surface to volumen ratio
 leads to a sensor with higher sensibility. The SnO$_{2}$  is usually known as a non-stoichiometric oxide, deficient in oxygen.
 The source of the sensibility to oxygen at the SnO$_{2}$  surface is attributed to the variable valence of Sn. 
A partial reduction of Sn$^{+2}$ to Sn$^{+4}$  can occur due to a presence of O$_{2}$ atmosphere \cite{Dobrovolskii1992} %%% buscar cual es la bibliografía



\end{document}
